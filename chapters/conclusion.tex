\chapter*{Conclusion}
Maintenance is an important process in all industrial facilities that ensures equipment availability and reduces equipment downtime. We can distinguish different types of maintenance actions where each type has a specific area of application. For critical equipment where repair cost is elevated and downtime can result in large losses, predictive maintenance was the preferred maintenance policy for many decades, yet with the technological advance in the last couple of decades where computational power become cheaper and sensors are ubiquitous, the industry started to shift towards what's called predictive maintenance and prognostics, which aim at providing predictions about the system's future behavior and potential breakdowns. There are many approaches to predictive maintenance and prognostics. This thesis focused on using data-driven methods, mainly machine learning and deep learning techniques and equipment monitoring data to develop predictive models that are able to learn degradation patterns of specific equipment and use it to predict the performance on new equipment. Chapter \ref{chapter:equipment_state_evaluation_using_neural_networks} used NASA C-MAPSS dataset to showcase the use of artificial neural networks (both fully-connected neural networks and LSTM networks) to model equipment degradation using data from sensors that monitor and measure different physical variables. The developed models showed great performance at predicting equipment remaining useful life (\acrshort{rul}). Later, the chapter presented the potential adoption of this approach in oilfields and on oil equipment.

Chapter \ref{chapter:diagnostic-and-prognostic-of-bearings-using-neural-networks} focused on more specialized application of neural networks to monitor bearings using vibration data for diagnostic and prognostic. The chapter used more sophisticated data processing techniques to extract suitable features from raw vibration data. Two different convolutional neural networks (\acrshort{cnn}) were  used, the first was used to diagnose bearings and classify fault type from vibration data, and the second used signal processing techniques and continuous wavelet transform to construct a dataset of scaleograms from vibration data which then was used as an input to the neural network to classify healthy and faulty bearings. The chapter also presented new feature extraction techniques from litarture using trigonometric features and cumulative descriptors along with metrics used to quantify the suitability of these features to prognostics. Finally, a procedure was described to adopt solutions introduced in this chapter to Top Drives in oilfields.
