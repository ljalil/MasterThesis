\chapter*{Conclusion}
\addcontentsline{toc}{chapter}{Conclusion}
La maintenance est un processus important dans toutes les installations industrielles qui garantit la disponibilité des équipements et réduit les temps d'arrêt. Nous pouvons distinguer différents types d'actions de maintenance, chaque type ayant un domaine d'application spécifique. Pour les équipements critiques dont le coût de réparation est élevé et dont les temps d'arrêt peuvent entraîner des pertes importantes, la maintenance prédictive a été la politique de maintenance privilégiée pendant de nombreuses décennies. Cependant, avec l'avancée technologique des deux dernières décennies où la puissance de calcul est devenue moins chère et les capteurs omniprésents, l'industrie a commencé à s'orienter vers ce que l'on appelle la maintenance prédictive et les pronostics, qui visent à fournir des prévisions sur le comportement futur du système et les pannes potentielles. Il existe de nombreuses approches de la maintenance prédictive et des pronostics. Cette thèse s'est concentrée sur l'utilisation de méthodes basées sur les données, principalement des techniques d'apprentissage machine et d'apprentissage profond et des données de surveillance des équipements pour développer des modèles prédictifs capables d'apprendre les modèles de dégradation d'un équipement spécifique et de les utiliser pour prédire les performances d'un nouvel équipement. Le chapitre \ref{chapter:equipment_state_evaluation_using_neural_networks} a utilisé la base de données C-MAPSS de la NASA pour illustrer l'utilisation des réseaux de neurones artificiels (tant les réseaux de neurones entièrement connectés que les réseaux LSTM) pour modéliser la dégradation des équipements à l'aide de données provenant de capteurs qui surveillent et mesurent différentes variables physiques. Les modèles développés ont montré une grande performance dans la prévision de la durée de vie utile restante des équipements (\acrshort{rul}). Plus tard, le chapitre a présenté l'adoption potentielle de cette approche dans les chantiers pétroliers sur l'équipement pétrolier.

Le chapitre \ref{chapter:diagnostic-and-prognostic-of-bearings-using-neural-networks} s'est concentré sur une application plus spécialisée des réseaux de neurones : la surveillance des roulements en utilisant les données de vibration pour le diagnostic et le pronostic. Ce chapitre a utilisé des techniques de traitement de données plus sophistiquées pour extraire les caractéristiques appropriées des données vibratoires brutes. Deux réseaux de neurones convolutionnels différents (\acrshort{cnn}) ont été utilisés, le premier pour diagnostiquer les roulements et classifier le type de défaut à partir des données de vibration, et le second pour utiliser des techniques de traitement du signal et de transformation en ondelettes continues afin de construire une base de données de scalogrammes à partir des données de vibration qui a ensuite été utilisée comme entrée dans le réseau de neurones pour classifier les roulements sains et les roulements défectueux. Le chapitre a également présenté de nouvelles techniques d'extraction de caractéristiques de la littérature en utilisant des caractéristiques trigonométriques et des descripteurs cumulatifs ainsi que des mesures utilisées pour quantifier l'adéquation de ces caractéristiques aux pronostics. Enfin, une procédure a été décrite pour adopter les solutions présentées dans ce chapitre aux entraînements de pointe dans les chantiers pétroliers.
