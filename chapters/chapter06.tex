\chapter{Diagnostic des Roulements à l'Aide de Réseaux de Neurones}

\chapterintrobox{La surveillance des vibrations est vitale pour de nombreux systèmes industriels. Les mesures des vibrations contiennent des informations très utiles sur l'état de santé des équipements et les types de défauts. Néanmoins, obtenir des informations à partir de ces signaux dans des applications réelles se révèle être un processus complexe. Cela est principalement dû à la complexité du problème. Ce chapitre présente une approche pour le diagnostic des défauts de roulements à l'aide de réseaux de neurones convolutifs.}

\section{Données de Case Western Reserve University}

\begin{wrapfigure}[14]{r}{0.5\textwidth}
    \centering
	\begin{tikzpicture}
	\node (outer) at (3.5,1.2) {\makecell{\small Outer\\Race}};
	\node (inner) at (3.5,0) {\makecell{\small Inner\\Race}};
	\node (ball) at (3.5,-1.2) {\makecell{\small Ball\\(in cage)}};
	
	\node (outer2) at (1.3,1.2) {};
	\node (inner2) at (.3,0) {};
	\node (ball2) at (-.5,-1.2) {};

	\node[inner sep=0] (image) at (0,0) {\includegraphics[width=0.35\textwidth]{figures/skf.jpg}};

	\draw [|->,  thick, red] (outer.west) -- (outer2);
	\draw [|->,  thick, red] (ball) -- (ball2);
	\draw [|->,  thick, red] (inner) -- (inner2);
\end{tikzpicture}
	\caption{Composants d'un roulement}
    \label{figure:skf-bearing-components}    
\end{wrapfigure}

Les données utilisées dans ce chapitre sont des données de vibrations des roulements fournies par Case Western Reserve University (CWRU). Les roulements utilisés dans le test sont des roulements à billes SKF. La figure \ref{figure:skf-bearing-components} montre les différents composants d'un roulement à billes standard. L'essai a été réalisé là où les roulements supportent l'arbre d'un moteur de 2 chevaux dans différentes conditions de charge. 

Les roulements d'essai possèdent des défauts ponctuels qui ont été introduits par électroérosion avec des diamètres de défaut de 0,18 mm, 0,36 mm, 0,53 mm, 0,71 mm et 1,02 mm. Ces défauts ont été introduits dans la bille et les chemins de roulement intérieurs et extérieurs du roulement. Des roulements SKF ont été utilisés pour les défauts de 0,18 mm, 0,36 mm et 0,53 mm, et des roulements équivalents NTN ont été utilisés pour les défauts de 0,71 mm et 1,02 mm. Le tableau \ref{tableau:cwru-bearings-specification} contient les dimensions des modèles de roulements SKF utilisés et les fréquences correspondantes (en multiples de RPM) associées aux différents types de défauts.

\begin{table}[H]
	\centering
	\begin{tabu}{cc|[1.5pt]cc}
		\tabucline[1.5pt]{-} 
		Dimension		&	Valeur (mm)	&	Défaut 			& Fréquence ($\times$RPM Hz)	\\
		\hline
		Diamètre intérieur	&	25.00		& Chemin intérieur 		& 5.4152\\
		Diamètre extérieur&	52.00		& Chemin extérieur 		& 3.5848 \\
		Epaisseur 		&	15.00		& Cage		& 0.3983 \\
		Diamètre du pas	&	08.03		& Bille	& 4.7135\\
		\tabucline[1.5pt]{-} 
	\end{tabu}
	\caption{Dimensions des roulements CWRU et les fréquences de défaults}
	\label{table:cwru-bearings-specification}
\end{table}

\section{Génération de données à partir de vibration}
Les données de vibration non traitées ne peuvent pas être utilisées directement pour entraîner un réseau de neurones. Ce chapitre utilise l'approche proposée dans \cite{Wen2018} pour convertir les données de vibration non traitées en images. La figure \ref{fig:cw_bearings_data_generation} montre le principe de génération de données où les signaux de vibration à une dimension sont convertis en matrices à deux dimensions (images) en transformant des morceaux de longueur 4096 en matrices de 64$\times$64.

\begin{figure}[h]
	\centering
	\includegraphics{figures/cw_bearings_data_generation.pdf}
	\caption{Génération de données par conversion du signal en images 64$\times$64}
	\label{fig:cw_bearings_data_generation}
\end{figure}

Comme mentionné précédemment, plusieurs essais ont été réalisés avec différents types de défauts de roulements (c'est-à-dire des défauts de billes, de chemins de roulement intérieurs et extérieurs) avec différents diamètres de défaut. Les signaux des différents tests sont transformés en images pour servir d'entrée à un réseau de neurones convolutif qui sera entraîné à classer les différents signaux dans les types de défauts correspondants et leurs diamètres. Les signaux sont répartis en 10 classes : trois types de défauts différents et pour chaque type de défaut, il y a trois diamètres de défaut différents, plus un signal de base normal qui appartient au roulement sain. La figure \ref{fig:bearings_faults_samples} montre quelques échantillons des signaux transformés des neuf différents types de défaut/diamètres : 

\begin{figure}[h]
    \centering
	\includegraphics{figures/cw_bearings_faults_samples.pdf}
    \caption{Signaux convertis de différents types de défauts}
    \label{fig:bearings_faults_samples}
\end{figure}

Les signaux des différents types de défauts ont des longueurs variables, ce qui se traduit par un nombre différent d'images synthétisées par type de défaut. Le nombre d'images par type de défaut (classe) est donné par l'équation \ref{equation:labels-per-classe} :

\begin{equation}
	N=floor \left(\frac{\text{signal length}}{64\times64}\right)
	\label{equation:labels-per-class}
\end{equation}

Le nombre d'images correspondant à chaque classe est indiqué dans la figure \ref{fig:bearings_faults_samples_count}:

\begin{figure}[h]
    \centering
	\includegraphics{figures/cw_bearings_faults_count.pdf}
    \caption{Converted signals of different faults types}
	\label{fig:bearings_faults_samples_count}
\end{figure}

%\begin{comment
\begin{table}[h]
	\centering
	\begin{tabu}{lc}
		\tabucline[1.5pt]{-} 
	   Class 					&	Samples count	\\
	   \hline 
	   Normal bearing 			&	295				\\
	   Roller element 0.18mm 	&	146				\\
	   Roller element 0.36mm 	&	116				\\
	   Roller element 0.54mm	&	116				\\
	   Inner race 0.18mm		&	295				\\
	   Inner race 0.36mm		&	116				\\
	   Inner race 0.54mm		&	116				\\
	   Outer race 0.18mm		&	116				\\
	   Outer race 0.36mm		&	116				\\
	   Outer race 0.54mm		&	116				\\
   \tabucline[1.5pt]{-}
   \end{tabu}
   \caption{}
   \label{table:cw-classes-count}
\end{table}
%\end{comment

\section{Diagnostic des défauts de roulements à l'aide de réseaux de neurones}
Après avoir généré des données en convertissant des signaux vibratoires bruts en images, ces images servent d'entrée pour un réseau de neurones convolutif.

\subsection{Architecture de réseau}
\acrlong{cnn} (\acrshort{cnn}) décrit un type de réseaux de neurones qui convient au traitement des images. Cette section présente l'utilisation de \acrshort{cnn} pour classer les signaux de vibration des roulements qui ont été transformés en images dans les types de défauts correspondants. Les \acrshort{cnn} ont été décrits en détail dans la section \ref{section:cnn}.

Pour effectuer cette tâche de classification, une architecture \acrshort{cnn} est utilisée, le réseau se compose de trois couches convolutionnelles avec des couches MaxPool entre elles, suivies de trois couches Fully-Connected. Tous les détails de cette architecture sont mentionnés dans le tableau\ref{table:bearings-faults-cnn-classifier-architecture}.

\begin{table}[h]
    \centering
    \begin{tabu}{lll}
		\tabucline[1.5pt]{-}
		\textbf{Couche (type)}   & \textbf{Forme de la sortie} &   \textbf{Param \#} \\
		\tabucline[1pt]{-}
		Conv1 (Conv2D) 			&   (None, 1, 64 ,32)   &   18464   \\
		MaxPool1 (MaxPool2D) 	&   (None, 1, 32, 32)   &   0       \\
		Conv2 (Conv2D)			&   (None, 1, 32, 64)   &   18496   \\
		MaxPool2 (MaxPooling2D) &   (None, 1, 16, 64)   &   0       \\
		Conv3 (Conv2D)          &   (None, 1, 16, 128)  &   73856   \\
		MaxPool3 (MaxPooling2D) &   (None, 1, 8, 128)   &   0       \\       
		Flatten1 (Flatten)      &   (None, 1024)        &   0       \\     
		Dense1 (Dense)          &   (None, 128)         &   131200  \\   
		Dense2 (Dense)          &   (None, 64)          &   8256    \\     
		Dense3 (Dense)          &   (None, 10)          &   650     \\
		\tabucline[1pt]{-}
		Total params: 250,922       &                   &           \\
		Trainable params: 250,922   &                   &           \\
		Non-trainable params: 0     &                   &           \\
	\tabucline[1.5pt]{-}
    \end{tabu}
    \caption{Architecture \acrshort{cnn}}
    \label{table:bearings-faults-cnn-classifier-architecture}
\end{table}


\subsection{Processus d'entraînement}
La base de données contient au total 1 548 échantillons, dont 20 \% sont utilisés comme test, les 80 \% restants étant destinés à l'entraînement. De l'ensemble d'entraînement, 15 \% sont utilisés comme fractionnement de validation pendant le processus d'entraînement. Le réseau a été entraîné pour 30 époques et batch size de 32.
La figure \ref{fig:bearings_faults_classification_training} montre l'évolution de la précision (\%) et de la perte (entropie croisée catégorique) en fonction des époques d'entraînement. Le réseau converge vers la 10éme époque avec une légère augmentation de la précision et une diminution de la perte par la suite. 

\begin{figure}[H]
    \centering
    \includegraphics{figures/cw_bearings_faults_classification_training.pdf}
    \caption{Entraînement de classificateur}
    \label{fig:bearings_faults_classification_training}
\end{figure}

\subsection{Discussion des résultats}
Le modèle atteint une précision parfaite de 100 \% sur l'ensemble d'entraînement et une précision proche sur l'ensemble de validation. Après l'entraînement, le modèle est évalué sur l'ensemble de test et atteint une précision presque parfaite de 98,72 \%. L'entraînement, la validation et la perte et la précision du test sont résumées dans le tableau \ref{tableau:cw-cnn-results}.

\begin{table}[H]
	\centering
	\begin{tabu}{lcc}
		\tabucline[1.5pt]{2-3} 
						&	\textbf{Perte}	&	\textbf{Précision}	\\
	   \tabucline[1pt]{-}
		Ensemble d'entraînement 		&	0.0003			&	100.00\%				\\
		Ensemble de validation 	&	0.0269 			&	99.46\%					\\
		Ensemble de test		&	0.0586 			&	98.71\%					\\
   \tabucline[1.5pt]{-}
   \end{tabu}
   \caption{Résultats de l'entraînement}
   \label{table:cw-cnn-results}
\end{table}

Pour mieux comprendre les résultats du modèle, une matrice de confusion est construite. La matrice de confusion (Figure \ref{fig:bearings_faults_classification_confusion_matrix}) montre les résultats des prédictions du modèle sur l'ensemble de test où l'axe des y montre les classes réelles (réelles) dans l'ensemble de test et l'axe des x les prédictions du modèle. La diagonale de la matrice de confusion est l'endroit où les classes réelles croisent les classes prédites par le modèle pour chaque échantillon et il est évident que le modèle atteint une classification quasi parfaite.

\begin{figure}[H]
    \centering
    \includegraphics{figures/cw_bearings_faults_classification.pdf}
    \caption{Matrice de confusion de la classification des défauts de roulements à l'aide de CNN}
    \label{fig:bearings_faults_classification_confusion_matrix}
\end{figure}

\section{Conclusion}
Ce chapitre a adopté une approche de génération de données provenant de la littérature qui convertit les signaux de vibration en images. Les signaux de vibration utilisés ici correspondent à des roulements présentant différents types de défauts et diamètres de défaut. Après avoir converti les signaux non traités en images, un \acrshort{cnn} est utilisé pour classer les signaux transformés (c'est-à-dire les images) dans leurs types de défauts et diamètres correspondants. Le modèle permet d'obtenir une précision de classification quasi parfaite sur l'ensemble de test.


\begin{comment}
\chapter{Bearings Faults Diagnostic using Neural Networks}

\chapterintrobox{Vibration condition monitoring are vital for many industrial systems, vibration data contains very useful information about health state of the equipment and the fault type. Nevertheless, gaining insights from vibration signals in real-world applications turns out to be a complex—and in many times, unfruitful—process. This is mainly due to the complexity of the problem. This chapter demonstrates image processing based approach for bearings faults diagnostics using convolutional neural networks.}

\section{Case Western Reserve University Bearings Data}

\begin{wrapfigure}{r}{0.5\textwidth}
    \centering
	\begin{tikzpicture}
	\node (outer) at (3.5,1.2) {\makecell{\small Outer\\Race}};
	\node (inner) at (3.5,0) {\makecell{\small Inner\\Race}};
	\node (ball) at (3.5,-1.2) {\makecell{\small Ball\\(in cage)}};
	
	\node (outer2) at (1.3,1.2) {};
	\node (inner2) at (.3,0) {};
	\node (ball2) at (-.5,-1.2) {};

	\node[inner sep=0] (image) at (0,0) {\includegraphics[width=0.35\textwidth]{figures/skf.jpg}};

	\draw [|->,  thick, red] (outer.west) -- (outer2);
	\draw [|->,  thick, red] (ball) -- (ball2);
	\draw [|->,  thick, red] (inner) -- (inner2);
\end{tikzpicture}
	\caption{Bearing's components}
    \label{figure:skf-bearing-components}    
\end{wrapfigure}
\vspace{-1em}

The dataset used in this chapter is bearings vibration dataset provided by Case Western Reserve University (CWRU). The bearings used in the test are SKF ball bearings. Figure \ref{figure:skf-bearing-components} shows the different components of a standard ball bearing.The test was conducted where the bearings support the shaft of a 2hp motor at different loading conditions. 

The test bearings have single point faults which were introduced using electro-discharge matching with fault diameters of 0.18mm, 0.36mm, 0.53mm, 0.71mm and 1.02mm. These faults were introduced in the bearing's ball, inner and outer raceways. SKF bearings were used for the 0.18mm, 0.36mm and 0.53mm faults, and NTN equivalent were used for the 0.71mm and 1.02mm faults. Table \ref{table:cwru-bearings-specification} contains the used SKF bearings' model dimensions and the corresponding frequencies (as multiples of RPM) associated with with different faults types.

\begin{table}[H]
	\centering
	\begin{tabu}{cc|[1.5pt]cc}
		\tabucline[1.5pt]{-} 
		Dimension		&	Size (mm)	&	Defect 			& Frequency ($\times$RPM Hz)	\\
		\hline
		Inner diameter	&	25.00		& Inner Ring 		& 5.4152\\
		Outside diameter&	52.00		& Outer Ring 		& 3.5848 \\
		Thickness 		&	15.00		& Cage Train		& 0.3983 \\
		Pitch diameter	&	08.03		& Rolling Element	& 4.7135\\
		\tabucline[1.5pt]{-} 
	\end{tabu}
	\caption{CWRU bearings dimensions and faults frequencies}
	\label{table:cwru-bearings-specification}
\end{table}

\section{Generating data from raw vibration signals}
Raw vibration data can't be used directly as an input to a neural network. This chapter uses the approach proposed in \cite{Wen2018} to convert raw vibration data into images. Figure \ref{fig:cw_bearings_data_generation} shows data generation principal where 1-dimensional vibration signals are converted into 2-dimensional arrays (images) by reshaping chunks of length 4096 into matrices of 64$\times$64.

\begin{figure}[h]
	\centering
	\includegraphics{figures/cw_bearings_data_generation.pdf}
	\caption{Data generation by converting the signal into 64$\times$64 images}
	\label{fig:cw_bearings_data_generation}
\end{figure}

As mentioned before, several tests have been conducted with different types of bearings faults (i.e. ball, inner and outer raceways faults) with different fault diameters. Signals from the different tests are transformed into images to serve as an input for a convolutional neural network which will be trained to classify the different signals into the corresponding fault types and their diameters. Signals are distributed across 10 classes: three different fault types and for each fault type there are three different fault diameters, plus a normal baseline signal that belongs to healthy bearing. Figure \ref{fig:bearings_faults_samples} shows some samples of the transformed signals of the different nine fault types/diameters: 

\begin{figure}[h]
    \centering
	\includegraphics{figures/cw_bearings_faults_samples.pdf}
    \caption{Converted signals of different faults types}
    \label{fig:bearings_faults_samples}
\end{figure}

Signals of the different fault types have varying lengths which results in a different number of synthesized images per fault type. Number of images per class is given by equation \ref{equation:labels-per-class}:

\begin{equation}
	N=floor \left(\frac{\text{signal length}}{64\times64}\right)
	\label{equation:labels-per-class}
\end{equation}

Number of images corresponding to each class is shown in in Figure \ref{fig:bearings_faults_samples_count}:

\begin{figure}[h]
    \centering
	\includegraphics{figures/cw_bearings_faults_count.pdf}
    \caption{Converted signals of different faults types}
	\label{fig:bearings_faults_samples_count}
\end{figure}

%\begin{comment
\begin{table}[h]
	\centering
	\begin{tabu}{lc}
		\tabucline[1.5pt]{-} 
	   Class 					&	Samples count	\\
	   \hline 
	   Normal bearing 			&	295				\\
	   Roller element 0.18mm 	&	146				\\
	   Roller element 0.36mm 	&	116				\\
	   Roller element 0.54mm	&	116				\\
	   Inner race 0.18mm		&	295				\\
	   Inner race 0.36mm		&	116				\\
	   Inner race 0.54mm		&	116				\\
	   Outer race 0.18mm		&	116				\\
	   Outer race 0.36mm		&	116				\\
	   Outer race 0.54mm		&	116				\\
   \tabucline[1.5pt]{-}
   \end{tabu}
   \caption{}
   \label{table:cw-classes-count}
\end{table}
%\end{comment

\section{Bearings faults diagnostic using neural networks}
After generating data by converting raw vibration signals into images, these images serve as an input for a convolutional neural network.

\subsection{Network architecture}
\acrlong{cnn} (\acrshort{cnn}) describes a type of neural networks that is suitable for image processing. This section presents the use of \acrshort{cnn} to classify bearings vibration signals that have been transformed into images into the corresponding fault types. \acrshort{cnn} were described in details in section \ref{section:cnn}.

To perform this classification task, a \acrshort{cnn} architecture is used, the network consists of three convolutional layers with max-pooling layers in between, followed by three fully-connected layers. All the details of this architecture from are mentioned in Table \ref{table:bearings-faults-cnn-classifier-architecture}.



\begin{table}[h]
    \centering
    \begin{tabu}{lll}
		\tabucline[1.5pt]{-}
		\textbf{Layer (type)}   & \textbf{Output shape} &   \textbf{Param \#} \\
		\tabucline[1pt]{-}
		Conv1 (Conv2D) 			&   (None, 1, 64 ,32)   &   18464   \\
		MaxPool1 (MaxPool2D) 	&   (None, 1, 32, 32)   &   0       \\
		Conv2 (Conv2D)			&   (None, 1, 32, 64)   &   18496   \\
		MaxPool2 (MaxPooling2D) &   (None, 1, 16, 64)   &   0       \\
		Conv3 (Conv2D)          &   (None, 1, 16, 128)  &   73856   \\
		MaxPool3 (MaxPooling2D) &   (None, 1, 8, 128)   &   0       \\       
		Flatten1 (Flatten)      &   (None, 1024)        &   0       \\     
		Dense1 (Dense)          &   (None, 128)         &   131200  \\   
		Dense2 (Dense)          &   (None, 64)          &   8256    \\     
		Dense3 (Dense)          &   (None, 10)          &   650     \\
		\tabucline[1pt]{-}
		Total params: 250,922       &                   &           \\
		Trainable params: 250,922   &                   &           \\
		Non-trainable params: 0     &                   &           \\
	\tabucline[1.5pt]{-}
    \end{tabu}
    \caption{\acrshort{cnn} architecture}
    \label{table:bearings-faults-cnn-classifier-architecture}
\end{table}


\subsection{Training process}
The dataset contains in total 1548 samples, 20\% is used as a test set while the remaining 80\% is the training set. From the train set, 15\% is used as a validation split during the training process. The network was trained for 30 epochs and a batch size of 32.
Figure \ref{fig:bearings_faults_classification_training} shows the evolution of the accuracy (\%) and loss (categorical cross-entropy) as a function of training epochs. The network converges around 10th epoch with minor increase in accuracy and decrease in loss afterwards. 

\begin{figure}[h]
    \centering
    \includegraphics{figures/cw_bearings_faults_classification_training.pdf}
    \caption{Classifier training}
    \label{fig:bearings_faults_classification_training}
\end{figure}

\subsection{Network results discussion}
The model achieves perfect 100\% accuracy on training set and a close accuracy on validation set. After training the model is evaluated on the test set and achieves a near perfect accuracy of 98.72\% accuracy. Train, validation and test loss and accuracy are summarized in Table \ref{table:cw-cnn-results}.

\begin{table}[H]
	\centering
	\begin{tabu}{lcc}
		\tabucline[1.5pt]{2-3} 
						&	\textbf{Loss}	&	\textbf{Accuracy}	\\
	   \tabucline[1pt]{-}
		Train set 		&	0.0003			&	100.00\%				\\
		Validation set 	&	0.0269 			&	99.46\%					\\
		Test set		&	0.0586 			&	98.71\%					\\
   \tabucline[1.5pt]{-}
   \end{tabu}
   \caption{\acrshort{cnn} training results}
   \label{table:cw-cnn-results}
\end{table}

To further understand the model's results, confusion matrix is constructed. The confusion matrix (Figure \ref{fig:bearings_faults_classification_confusion_matrix}) shows the results of model predictions on the test set where y-axis shows the actual (real) labels in the test set and x-axis the model predictions. Diagnoal of confusion matrix is where actual labels intersect with model's predicted labels for each sample and it is apparent that the model achieves near-perfect classification.

\begin{figure}[h]
    \centering
    \includegraphics{figures/cw_bearings_faults_classification.pdf}
    \caption{Confusion matrix of bearings faults classification using \acrshort{cnn}}
    \label{fig:bearings_faults_classification_confusion_matrix}
\end{figure}

\section{Conclusion}
This chapter adopted a data generation approach from literature that converts vibration signals into images. Vibration signals used here corresponds to bearings with different types of faults and fault diameters. After converting raw signals into images, a \acrshort{cnn} is used to classify transformed signals (i.e. images) into their corresponding faults types and diameters. The model achieves near-perfect classification accuracy on the test set.
\end{comment}